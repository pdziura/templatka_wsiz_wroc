% załącznik nr 6
% Made by Mateusz Cedro
\addcontentsline{toc}{chapter}{Oświadczenie autorskie} % dodatkowy wpis do spisu treści

\makeatletter
\newgeometry{tmargin=3.3cm,bmargin=2.0cm,lmargin=2.0cm,rmargin=2.0cm,bindingoffset=1.0cm,nomarginpar,nohead}

\begin{flushright}
    Wrocław, dnia \@when
  \end{flushright}

  \begin{flushleft}
    {\fontsize{14}{14} \selectfont
    
    \vspace*{20pt}
    {\bf Wydział Informatyki}
    
    \bigskip
    Kierunek studiów: {\bf informatyka (INF)}
    }  
  \vspace*{30pt}
  
  {~\@author} % autor pracy
  \vspace*{-4pt}
  
  \makebox[220pt][r]{\dotfill} %zastępuję dotline

  \vspace*{-2pt}
  {\scriptsize (imię i nazwisko studenta)}

  \bigskip
  {~\@album} %nr indeksu
  
  \vspace*{-4pt}
  \makebox[220pt][r]{\dotfill} %zastępuję dotline
  
  \vspace*{-2pt}
  {\scriptsize (nr albumu)}
\end{flushleft}

  \bigskip

\begin{center}
  {\fontsize{16}{16} \selectfont \bf OŚWIADCZENIE AUTORSKIE}
\end{center}

\bigskip


    Oświadczam, że niniejszą pracę dyplomową pod tytułem:
    \begin{center}
        {\fontsize{16}{16} \selectfont \bf \@title}
    \end{center}
    napisałem/am samodzielnie. Nie korzystałem/am z pomocy osób trzecich, jak również nie dokonałem/am zapożyczeń z innych prac.
    \newline
    \newline
    \indent Wszystkie fragmenty pracy takie jak cytaty, ryciny, tabele, programy itp., które nie są mojego autorstwa, zostały odpowiednio zaznaczone i zamieszczono w pracy źródła ich pochodzenia. Treść wydrukowanej pracy dyplomowej jest identyczna z wersją pracy zapisaną na przekazywanym nośniku elektronicznym.
    \newline
    \newline
    \indent Jednocześnie przyjmuję do wiadomości, że jeżeli w przypadku postępowania wyjaśniającego zebrany materiał potwierdzi popełnienie przeze mnie plagiatu, skutkować to będzie niedopuszczeniem do dalszych czynności w sprawie nadania mi tytułu zawodowego do czasu wydania orzeczenia przez komisję dyscyplinarną oraz złożenie zawiadomienia o popełnieniu przestępstwa.

\bigskip

\vspace*{30pt}
\begin{flushright} % wyrownanie do prawej
    \makebox[150pt]{\dotfill} % wypełniamy kropkami box
  
  \vspace*{-2pt}
  \makebox[150pt]{\scriptsize (podpis studenta)} % nowy box z podpisem
\end{flushright}

\restoregeometry
\makeatother