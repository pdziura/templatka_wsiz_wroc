\documentclass[12pt]{bsc}

\usepackage[T1,MeX]{polski}
\usepackage[utf8]{inputenc}
\usepackage[T1]{fontenc}
\usepackage{txfonts}

\usepackage{graphicx}
\usepackage{fancyvrb}
\usepackage{pdfpages}

\usepackage{enumitem}
\usepackage{listings}
\usepackage{caption}

\usepackage{alltt}

%sposob na zablokowanie obrazka na stronie
%\FloatBarrier (wcześniej dołącz \usepackage{placeins})
\usepackage{placeins}


%caption listingów
\DeclareCaptionFont{black}{ \color{black} }
\DeclareCaptionFormat{listing}{
  {
    \parbox{\textwidth}{\hspace{12pt}#1#2#3}
  }
}
\captionsetup[lstlisting]{ format=listing, labelfont=black, textfont=black, singlelinecheck=false, margin=0pt, font={bf,footnotesize} }

%definicja custom listingów
\lstdefinestyle{custom}{
  breaklines=true,
  xleftmargin=\parindent,
  showstringspaces=false,
  basicstyle=\fontsize{10}{11}\selectfont,
}

%pozwala na używanie znaku dolara($) w listingach
\lstset{
    mathescape=false
}

\usepackage[pdftex,pdfstartview=FitH,unicode]{hyperref}
% "klikalny" spis treści
\hypersetup{
    colorlinks,
    citecolor=black,
    filecolor=black,
    linkcolor=black,
    urlcolor=black
}

%https://tex.stackexchange.com/questions/36880/insert-a-blank-page-after-current-page
\usepackage{afterpage}

\newcommand\blankpage{%
    \null
    \thispagestyle{empty}%
    \addtocounter{page}{-1}%
    \newpage}
    


\graphicspath{{./img/}}

\selecthyphenation{polish}


% ********* documents meta data **********
\makeatletter

\def\@title{nazwa pracy}
\def\@engTitle{ang nazwa pracy}
\def\@author{Imie i nazwisklo}
\def\@promoter{tytul imie nazwisko}
\def\@when{2020-10-10}
\def\@year{2021}
\def\@album{9999}

\makeatother

\renewcommand{\partname}{}

\begin{document}
  \include{storna_glowna}
  \include{pusta}
  \tableofcontents
  \cleardoublepage
\phantomsection
\addcontentsline{toc}{chapter}{Słownik pojęć}
\chapter*{Słownik pojęć}
\textbf{Przykladowe pojecia}
  \include{wstep}
  
  
  \part{Część przeglądowa}  
  \include{przegladowa}
  
 
  \part{Część praktyczna}
  \include{praktyczna}
  \part{Testy}
  \include{testy}
 
 
  \part{Podsumowanie}
  \include{podsumowanie}
  
  
  \pagestyle{plain}
  \cleardoublepage
\phantomsection
\addcontentsline{toc}{chapter}{Bibliografia}

\begin{thebibliography}{99}





%przykladowy wpis:
 \bibitem{nazwa} % ta nazwa to jest odniesienie do przypisu harwardzkiego
	Example: 
	https://example.com - odczytano dnia 27.10.2020

\end{thebibliography}


  
  \cleardoublepage
\phantomsection
\addcontentsline{toc}{chapter}{Spis rysunków}
\listoffigures

%\cleardoublepage
%\phantomsection
%\addcontentsline{toc}{chapter}{Spis tabel}
%\listoftables

\cleardoublepage
\phantomsection
\addcontentsline{toc}{chapter}{Spis listingów}
\renewcommand{\lstlistlistingname}{Spis listingów}
\lstlistoflistings


%\cleardoublepage
%\phantomsection
%\addcontentsline{toc}{chapter}{Dodatki}
%\setlist{Dodatki}



%czyści puste strony
\let\cleardoublepage\clearpage

  \addcontentsline{toc}{chapter}{Załączniki}
\chapter*{Załączniki}
Załącznik 1 - zawartość skryptu skrypt.sh\
\lstinputlisting[style=custom,nolol]{sample_script.sh}
  \cleardoublepage
\newgeometry{tmargin=3.3cm,bmargin=2.0cm,lmargin=2.0cm,rmargin=2.0cm,bindingoffset=1.0cm,nomarginpar,nohead}
\thispagestyle{plain}

\phantomsection
\addcontentsline{toc}{chapter}{Zawartość płyty CD}

Zawartość płyty CD
\begin{enumerate}
\item praca\_inzynierksa.pdf;
\item Plik z kodem źródłowym: sample\_script.sh;
\end{enumerate}

  \include{Oswiadczenie_udostepnienie}
  \include{Oswiadczenie_autorskie}
\end{document}