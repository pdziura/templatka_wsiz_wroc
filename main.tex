\documentclass[12pt]{bsc}

\usepackage[T1,MeX]{polski}
\usepackage[utf8]{inputenc}
\usepackage[T1]{fontenc}
\usepackage{txfonts}

\usepackage{graphicx}
\usepackage{fancyvrb}
\usepackage{pdfpages}

\usepackage{enumitem}
\usepackage{listings}
\usepackage{caption}

\usepackage{alltt}

%sposob na zablokowanie obrazka na stronie
%\FloatBarrier (wcześniej dołącz \usepackage{placeins})
\usepackage{placeins}


%caption listingów
\DeclareCaptionFont{black}{ \color{black} }
\DeclareCaptionFormat{listing}{
  {
    \parbox{\textwidth}{\hspace{12pt}#1#2#3}
  }
}
\captionsetup[lstlisting]{ format=listing, labelfont=black, textfont=black, singlelinecheck=false, margin=0pt, font={bf,footnotesize} }

%definicja custom listingów
\lstdefinestyle{custom}{
  breaklines=true,
  xleftmargin=\parindent,
  showstringspaces=false,
  basicstyle=\fontsize{10}{11}\selectfont,
}

%pozwala na używanie znaku dolara($) w listingach
\lstset{
    mathescape=false
}

\usepackage[pdftex,pdfstartview=FitH,unicode]{hyperref}
% "klikalny" spis treści
\hypersetup{
    colorlinks,
    citecolor=black,
    filecolor=black,
    linkcolor=black,
    urlcolor=black
}

%https://tex.stackexchange.com/questions/36880/insert-a-blank-page-after-current-page
\usepackage{afterpage}

\newcommand\blankpage{%
    \null
    \thispagestyle{empty}%
    \addtocounter{page}{-1}%
    \newpage}
    


\graphicspath{{./img/}}

\selecthyphenation{polish}


% ********* documents meta data **********
\makeatletter

\def\@title{nazwa pracy}
\def\@engTitle{ang nazwa pracy}
\def\@author{Imie i nazwisklo}
\def\@promoter{tytul imie nazwisko}
\def\@when{2020-10-10}
\def\@year{2021}
\def\@album{9999}

\makeatother

\renewcommand{\partname}{}

\begin{document}
  \thispagestyle{empty}

\makeatletter

\begin{titlepage}
  \sffamily\bfseries
  
  \center{
    \fontsize{18}{18}\selectfont{
      WYŻSZA SZKOŁA INFORMATYKI I ZARZĄDZANIA
      ,,COPERNICUS'' WE WROCŁAWIU
    }
    \rule[10pt]{\textwidth}{1pt}
    \raisebox{12pt}[0px]{
      \fontsize{16}{16}\selectfont{WYDZIAŁ INFORMATYKI}
    }
  }
  
  \flushleft\fontsize{14}{14}\selectfont{
    \parbox{110pt}{\textmd{Kierunek studiów:}}
    Informatyka

    \parbox{110pt}{\textmd{Poziom studiów:}}
    Studia pierwszego stopnia-inżynierskie
    
    \parbox{110pt}{\textmd{Specjalność:}}
    Systemy i sieci komputerowe
  }
  
  \vspace*{40pt}
  \center{
    \fontsize{14}{14}\selectfont{PRACA DYPLOMOWA INŻYNIERSKA}
    
    \vspace*{30pt}
    \fontsize{14}{14}\selectfont\@author
    
    \vspace*{15pt}
    \fontsize{20}{20}\selectfont\@title
    
    \vspace*{20pt}
    \fontsize{14}{14}\selectfont\@engTitle
  }

  \vspace*{80pt}
  \flushright{  
    \fontsize{14}{14}\selectfont{Ocena pracy:}
    \makebox[220pt][r]{\fontsize{10}{10}\selectfont\dotfill}
    
    \fontsize{10}{10}\selectfont\textmd{(ocena pracy dyplomowej, data, podpis promotora) }
    
    \vspace*{50pt}
    \makebox[220pt][r]{\fontsize{10}{10}\selectfont\dotfill}
    
    \raisebox{4pt}{\fontsize{10}{10}\selectfont\textmd{(pieczątka uczelni)}}
  }
    
  \flushleft\fontsize{14}{14}\selectfont{
    Promotor:
    
    \bigskip\@promoter
  }

  \vspace*{20pt}
  \center{
    \rule[3pt]{\textwidth}{1pt}
    \fontsize{16}{16}\selectfont{WROCŁAW \@year}
  }
\end{titlepage}

\makeatother
\clearpage
%\newpage
%\thispagestyle{empty}
%\mbox{}
%\newpage

  \newpage
\thispagestyle{empty}
\mbox{}
\newpage
  \tableofcontents
  \cleardoublepage
\phantomsection
\addcontentsline{toc}{chapter}{Słownik pojęć}
\chapter*{Słownik pojęć}
\textbf{Przykladowe pojecia}
  \chapter{Wstęp}

\section{Wprowadzenie} %section tworzymy nowe sekcje np. 2.1
"Lorem ipsum dolor sit amet, consectetur adipiscing elit, sed do eiusmod tempor incididunt ut labore et dolore magna aliqua. Ut enim ad minim veniam, quis nostrud exercitation ullamco laboris nisi ut aliquip ex ea commodo consequat. Duis aute irure dolor in reprehenderit in voluptate velit esse cillum dolore eu fugiat nulla pariatur. Excepteur sint occaecat cupidatat non proident, sunt in culpa qui officia deserunt mollit anim id est laborum."
\section{Cel pracy}
"Lorem ipsum dolor sit amet, consectetur adipiscing elit, sed do eiusmod tempor incididunt ut labore et dolore magna aliqua. Ut enim ad minim veniam, quis nostrud exercitation ullamco laboris nisi ut aliquip ex ea commodo consequat. Duis aute irure dolor in reprehenderit in voluptate velit esse cillum dolore eu fugiat nulla pariatur. Excepteur sint occaecat cupidatat non proident, sunt in culpa qui officia deserunt mollit anim id est laborum."

\section{Zakres}
"Lorem ipsum dolor sit amet, consectetur adipiscing elit, sed do eiusmod tempor incididunt ut labore et dolore magna aliqua. Ut enim ad minim veniam, quis nostrud exercitation ullamco laboris nisi ut aliquip ex ea commodo consequat. Duis aute irure dolor in reprehenderit in voluptate velit esse cillum dolore eu fugiat nulla pariatur. Excepteur sint occaecat cupidatat non proident, sunt in culpa qui officia deserunt mollit anim id est laborum."

\subsection{Zakres2} %tak tworzymu subsekcje np. 2.3.1
"Lorem ipsum dolor sit amet, consectetur adipiscing elit, sed do eiusmod tempor incididunt ut labore et dolore magna aliqua. Ut enim ad minim veniam, quis nostrud exercitation ullamco laboris nisi ut aliquip ex ea commodo consequat. Duis aute irure dolor in reprehenderit in voluptate velit esse cillum dolore eu fugiat nulla pariatur. Excepteur sint occaecat cupidatat non proident, sunt in culpa qui officia deserunt mollit anim id est laborum."
  
  
  \part{Część przeglądowa}  
  \chapter{Część przeglądowa}

\section{Jakaś sekcja}
"Lorem ipsum dolor sit amet, consectetur adipiscing elit, sed do eiusmod tempor incididunt ut labore et dolore magna aliqua. Ut enim ad minim veniam, quis nostrud exercitation ullamco laboris nisi ut aliquip ex ea commodo consequat. Duis aute irure dolor in reprehenderit in voluptate velit esse cillum dolore eu fugiat nulla pariatur. Excepteur sint occaecat cupidatat non proident, sunt in culpa qui officia deserunt mollit anim id est laborum."




  
 
  \part{Część praktyczna}
  \chapter{Część praktyczna}

\section{jakas seckcja}
"Lorem ipsum dolor sit amet, consectetur adipiscing elit, sed do eiusmod tempor incididunt ut labore et dolore magna aliqua. Ut enim ad minim veniam, quis nostrud exercitation ullamco laboris nisi ut aliquip ex ea commodo consequat. Duis aute irure dolor in reprehenderit in voluptate velit esse cillum dolore eu fugiat nulla pariatur. Excepteur sint occaecat cupidatat non proident, sunt in culpa qui officia deserunt mollit anim id est laborum."
\let\cleardoublepage\clearpage
  \part{Testy}
  \chapter{Testy}
\section{Testy}
"Lorem ipsum dolor sit amet, consectetur adipiscing elit, sed do eiusmod tempor incididunt ut labore et dolore magna aliqua. Ut enim ad minim veniam, quis nostrud exercitation ullamco laboris nisi ut aliquip ex ea commodo consequat. Duis aute irure dolor in reprehenderit in voluptate velit esse cillum dolore eu fugiat nulla pariatur. Excepteur sint occaecat cupidatat non proident, sunt in culpa qui officia deserunt mollit anim id est laborum."


 
 
  \part{Podsumowanie}
  \chapter{Podsumowanie}
"Lorem ipsum dolor sit amet, consectetur adipiscing elit, sed do eiusmod tempor incididunt ut labore et dolore magna aliqua. Ut enim ad minim veniam, quis nostrud exercitation ullamco laboris nisi ut aliquip ex ea commodo consequat. Duis aute irure dolor in reprehenderit in voluptate velit esse cillum dolore eu fugiat nulla pariatur. Excepteur sint occaecat cupidatat non proident, sunt in culpa qui officia deserunt mollit anim id est laborum."

\chapter{Możliwości dalszego rozwoju}
"Lorem ipsum dolor sit amet, consectetur adipiscing elit, sed do eiusmod tempor incididunt ut labore et dolore magna aliqua. Ut enim ad minim veniam, quis nostrud exercitation ullamco laboris nisi ut aliquip ex ea commodo consequat. Duis aute irure dolor in reprehenderit in voluptate velit esse cillum dolore eu fugiat nulla pariatur. Excepteur sint occaecat cupidatat non proident, sunt in culpa qui officia deserunt mollit anim id est laborum."

\chapter{Zakończenie}
"Lorem ipsum dolor sit amet, consectetur adipiscing elit, sed do eiusmod tempor incididunt ut labore et dolore magna aliqua. Ut enim ad minim veniam, quis nostrud exercitation ullamco laboris nisi ut aliquip ex ea commodo consequat. Duis aute irure dolor in reprehenderit in voluptate velit esse cillum dolore eu fugiat nulla pariatur. Excepteur sint occaecat cupidatat non proident, sunt in culpa qui officia deserunt mollit anim id est laborum."
  
  
  \pagestyle{plain}
  \cleardoublepage
\phantomsection
\addcontentsline{toc}{chapter}{Bibliografia}

\begin{thebibliography}{99}





%przykladowy wpis:
 \bibitem{nazwa} % ta nazwa to jest odniesienie do przypisu harwardzkiego
	Example: 
	https://example.com - odczytano dnia 27.10.2020

\end{thebibliography}


  
  \cleardoublepage
\phantomsection
\addcontentsline{toc}{chapter}{Spis rysunków}
\listoffigures

%\cleardoublepage
%\phantomsection
%\addcontentsline{toc}{chapter}{Spis tabel}
%\listoftables

\cleardoublepage
\phantomsection
\addcontentsline{toc}{chapter}{Spis listingów}
\renewcommand{\lstlistlistingname}{Spis listingów}
\lstlistoflistings


%\cleardoublepage
%\phantomsection
%\addcontentsline{toc}{chapter}{Dodatki}
%\setlist{Dodatki}



%czyści puste strony
\let\cleardoublepage\clearpage

  \addcontentsline{toc}{chapter}{Załączniki}
\chapter*{Załączniki}
Załącznik 1 - zawartość skryptu skrypt.sh\
\lstinputlisting[style=custom,nolol]{sample_script.sh}
  \cleardoublepage
\newgeometry{tmargin=3.3cm,bmargin=2.0cm,lmargin=2.0cm,rmargin=2.0cm,bindingoffset=1.0cm,nomarginpar,nohead}
\thispagestyle{plain}

\phantomsection
\addcontentsline{toc}{chapter}{Zawartość płyty CD}

Zawartość płyty CD
\begin{enumerate}
\item praca\_inzynierksa.pdf;
\item Plik z kodem źródłowym: sample\_script.sh;
\end{enumerate}

  % załącznik nr 5
% Made by Mateusz Cedro

\addcontentsline{toc}{chapter}{Oświadczenie o udostępnianiu pracy dyplomowej} % dodanie do spisu treści

\makeatletter
\newgeometry{tmargin=3.3cm,bmargin=2.0cm,lmargin=2.0cm,rmargin=2.0cm,bindingoffset=1.0cm,nomarginpar,nohead}

\begin{flushright}
    Wrocław, dnia \@when
  \end{flushright}

  \begin{flushleft}
    {\fontsize{14}{14} \selectfont
    
    \vspace*{20pt}
    {\bf Wydział Informatyki}
    
    \bigskip
    Kierunek studiów: {\bf informatyka (INF)}
    }  
  \vspace*{30pt}
  
  {~\@author} % autor pracy
  \vspace*{-4pt}
  
  \makebox[220pt][r]{\dotfill} %zastępuję dotline

  \vspace*{-2pt}
  {\scriptsize (imię i nazwisko studenta)}

  \bigskip
  {~\@album} %nr indeksu
  
  \vspace*{-4pt}
  \makebox[220pt][r]{\dotfill} %zastępuję dotline
  
  \vspace*{-2pt}
  {\scriptsize (nr albumu)}
\end{flushleft}

  \bigskip

\begin{center}
  {\fontsize{16}{16} \selectfont \bf OŚWIADCZENIE O UDOSTĘPNIANIU PRACY DYPLOMOWEJ}
\end{center}

\bigskip

\begin{flushright} %parbox potrafi dzielić linie. Mbox - nie.
    \vspace*{-4pt}
    \parbox{350pt}{
        Tytuł pracy dyplomowej: ~\@title
      }
\end{flushright}

\bigskip

\begin{flushright}
  Wyrażam zgodę (nie wyrażam zgody)\footnote{Niepotrzebne skreślić.} na udostępnianie mojej pracy dyplomowej.
\end{flushright}

\bigskip

\vspace*{30pt}
\begin{flushright} % wyrownanie do prawej
    \makebox[150pt]{\dotfill} % wypełniamy kropkami box
  
  \vspace*{-2pt}
  \makebox[150pt]{\scriptsize (podpis studenta)} % nowy box z podpisem
\end{flushright}

\restoregeometry
\makeatother
  % załącznik nr 6
% Made by Mateusz Cedro
\addcontentsline{toc}{chapter}{Oświadczenie autorskie} % dodatkowy wpis do spisu treści

\makeatletter
\newgeometry{tmargin=3.3cm,bmargin=2.0cm,lmargin=2.0cm,rmargin=2.0cm,bindingoffset=1.0cm,nomarginpar,nohead}

\begin{flushright}
    Wrocław, dnia \@when
  \end{flushright}

  \begin{flushleft}
    {\fontsize{14}{14} \selectfont
    
    \vspace*{20pt}
    {\bf Wydział Informatyki}
    
    \bigskip
    Kierunek studiów: {\bf informatyka (INF)}
    }  
  \vspace*{30pt}
  
  {~\@author} % autor pracy
  \vspace*{-4pt}
  
  \makebox[220pt][r]{\dotfill} %zastępuję dotline

  \vspace*{-2pt}
  {\scriptsize (imię i nazwisko studenta)}

  \bigskip
  {~\@album} %nr indeksu
  
  \vspace*{-4pt}
  \makebox[220pt][r]{\dotfill} %zastępuję dotline
  
  \vspace*{-2pt}
  {\scriptsize (nr albumu)}
\end{flushleft}

  \bigskip

\begin{center}
  {\fontsize{16}{16} \selectfont \bf OŚWIADCZENIE AUTORSKIE}
\end{center}

\bigskip


    Oświadczam, że niniejszą pracę dyplomową pod tytułem:
    \begin{center}
        {\fontsize{16}{16} \selectfont \bf \@title}
    \end{center}
    napisałem/am samodzielnie. Nie korzystałem/am z pomocy osób trzecich, jak również nie dokonałem/am zapożyczeń z innych prac.
    \newline
    \newline
    \indent Wszystkie fragmenty pracy takie jak cytaty, ryciny, tabele, programy itp., które nie są mojego autorstwa, zostały odpowiednio zaznaczone i zamieszczono w pracy źródła ich pochodzenia. Treść wydrukowanej pracy dyplomowej jest identyczna z wersją pracy zapisaną na przekazywanym nośniku elektronicznym.
    \newline
    \newline
    \indent Jednocześnie przyjmuję do wiadomości, że jeżeli w przypadku postępowania wyjaśniającego zebrany materiał potwierdzi popełnienie przeze mnie plagiatu, skutkować to będzie niedopuszczeniem do dalszych czynności w sprawie nadania mi tytułu zawodowego do czasu wydania orzeczenia przez komisję dyscyplinarną oraz złożenie zawiadomienia o popełnieniu przestępstwa.

\bigskip

\vspace*{30pt}
\begin{flushright} % wyrownanie do prawej
    \makebox[150pt]{\dotfill} % wypełniamy kropkami box
  
  \vspace*{-2pt}
  \makebox[150pt]{\scriptsize (podpis studenta)} % nowy box z podpisem
\end{flushright}

\restoregeometry
\makeatother
\end{document}